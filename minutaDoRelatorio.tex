% Minuta para relatórios finais para a unidade curricular
% de "Estagio ou Projecto" do curso de Licenciatura em Engenharia Informática
% do Instituto Politécnico de Beja
% Versão de 2013/04/18
% Autor: João Paulo Barros, joao.barros@ipbeja.pt

\documentclass{ipbeja-trabalhos-academicos}
% Para preencher 

\newcommand{\ESCOLA}{Escola Superior de Tecnologia e Gestão}

\newcommand{\TITULO}{Business Intelligence - Open Source Driven}
\newcommand{\SUBTITULO}{Opcionalmente, colocar aqui sub-título com máximo de vinte palavras}
\newcommand{\TITLE}{Business Intelligence - Open Source Driven}
\newcommand{\SUBTITLE}{Put here the subtitle in english}


\newcommand{\CANDIDATO}{Emanuel Alexandre Cavaco Teixeira}

 %se for um projecto comentar a linha seguinte. Caso contrario indicar orientador
 % na entidade de acolhimento do estágio
\newcommand{\ORIENTADORENTIDADE}{Eng. Nuno Miguel Lopes dos Santos, Deloitte}
\newcommand{\ORIENTADORIPBA}{Nome completo do docente orientador e respectivo título académico}
% se não existir segundo orientador do IPBeja, comentar a linha seguinte
\newcommand{\ORIENTADORIPBB}{Nome completo do segundo docente orientador respectivo título académico} 


%Completar e comentar um dos seguintes dois \newcommand
%\newcommand{\DECLARACAOPROJETO}{Relatório de projeto de fim de curso apresentado na\linebreak \ESCOLA{} do Instituto Politécnico de Beja}
\newcommand{\DECLARACAOESTAGIO}{
Relatório de estágio, realizado na Deloitte, apresentado na\linebreak \ESCOLA{} do Instituto Politécnico de Beja}

% commentar se não existente
\newcommand{\DEDICATORIA}{dedication text}


\begin{document}
\folhacapa % 2.1 das normas
\folharosto % 2.3 das normas
%%%%%%%%%%%%%%%%%%%%%%%%%%%%%%%%%%%%%%%%%%%%
\frontmatter % parte inicial

\chapter{Resumo}
\section*{\textit{\TITULO}\\  {\small{\textit{\SUBTITULO}}}}

\par Este documento serve para descrever o trabalho realizado no âmbito da unidade curricular de Estágio e Projecto da Licenciatura de Engenharia Informática do Instituto Politécnico de Beja. O trabalho realizado consistia na construção de dashboards e gráficos sobre informações de gestão sobre receitas de tráfego, afectas ao ramo de negocio da empresa xpto. Esta informação será disponibilizada sob a forma de widgets para posteriormente serem incluídos no portal corporativo já existente.
\par A implementação destas funcionalidades foi efectuada através de tecnologias totalmente open source, fazendo com que esta informação chegue a um numero considerável de utilizadores, sem aumentar os custos de licenciamento inerentes a sistemas de reporting actualmente em uso na empresa xpto. Paralelamente a informação dos dashboards foi complementada com a inclusão de novas métricas e indicadores recorrendo ao QlikView. Isto permite uma visualização mais detalhada da informação, contudo menos acessível, devido aos custos de licenciamento.
\par Para além das tarefas supracitadas foi também desenvolvido um relatório em formato de newsletter na aplicação de report QlikView, e todo o processo de desenvolvimento estará descrito no presente documento.\newline

\textbf{Palavras-chave}: \textit{JavaScript, D3.js, AngularJS, JAVA, QlikView}.
\chapter{Abstract}
\section*{\textit{\TITLE}\\  {\small{\textit{\SUBTITLE}}}}

\textit{Between 100 and 200 words.}

...

...



\textbf{Keywords}: \textit{Specify between 5 and 10 keywords, separated by commas, about the theme of the report}.

% Se pretender remover, comente a linha seguinte, caso contrário preencha o ficheiro agradecimentos.tex
\input{parte-inicial/agradecimentos}


\indicegeral  
\indicedefiguras % só se existirem mais do que 5 figuras. Caso contrário, remova.
\indicedetabelas % só se existirem mais do que 5 tabelas. Caso contrário, remova.
\indicedelistagens % só se existirem mais do que 5 listagens

% No caso de se verificar "um número significativamente elevado de abreviaturas e siglas" deve retirar-se o 
% comentário da linha seguinte e preencher o ficheiro parte-inicial/abreviaturas.tex
%\chapter{Abreviaturas e Siglas}
\begin{quote} % para uma pequena indentação
\begin{tabular}[t]{p{4cm} p{10cm}}


IPBeja & Instituto Politécnico de Beja\\
UML & Unified Modelling Language\\
....... & .....



\end{tabular}

\end{quote}
 

%%%%%%%%%%%%%%%%%%%%%%%%%%%%%%%%%%%%%%%%%%%%
\mainmatter  \pagestyle{ruled} % parte principal

\chapter{Introdução}
\label{intro}

\section{Âmbito}

\par O projecto foi desenvolvido na empresa de auditoria e consultoria Deloitte, nas instalações da empresa XPTO, cliente da Deloitte, inserido no decorrer da unidade curricular de Estágio ou Projecto, integrante do plano de estudos do curso de Licenciatura em Engenharia Informática da ESTIG, estabelecimento de ensino pertencente ao IPBeja. O planeamento de estágio encontra um nível de dificuldade e complexidade adequado ao mercado de trabalho e face as funções que se pretende que um recém-licenciado na área da engenharia informática realize. O estágio realizado obedeceu às seguintes condições previamente estabelecidas:
\begin{itemize}  \itemsep1pt \parskip0pt \parsep0pt
 \item Ter uma duração de três meses, com uma carga horária que se assemelhasse ao verdadeiro mercado de trabalho, cerca de 8 horas diárias, totalizando um total de quarenta horas semanais.
 \item Ser orientado por uma figura de mérito reconhecido dentro da instituição, bem como um docente pertencente à ESTIG com o intuito de supervisionar o trabalho realizado na instituição de acolhimento. Na instituição de acolhimento pude contar com o Eng.º Nuno Santos, Senior Manager da área de AMS. A Professora Doutora Isabel Sofia foi a docente responsável por acompanhar o meu trabalho.
\item Obter previamente aprovação para o planeamento do estágio a que me propunha, por parte da Comissão de Estágios da LEI.
\end{itemize}
\pagebreak

\section{Instituição de acolhimento}

 A Deloitte surgiu em 1845 em Londres, sendo hoje uma das maiores empresas no mundo a operar no sector da auditoria, consultoria e outsourcing.
\par A marca “Deloitte” está espalhada pelo mundo, estando presente em mais de 150 países, com cerca de 700 escritórios e mais de 210.000 (duzentos e dez mil) colaboradores. Sobre a marca “Deloitte” operam firmas independentes, membro da Deloitte Touche Tohmatsu Limited, uma sociedade privada de responsabilidade limitada do Reino Unido. Cada firma pertencente à DTTL presta serviços numa determinada área geográfica e está restrita à legislação dessa mesma área geográfica onde opera.
\par Em Portugal a entidade membro da Deloitte é a Deloitte \& Associados, SROC S.A. Desta entidade legal fazem parte três subsidiarias que operam em ramos de negocio distinto, sendo que este projecto foi realizado ao serviço da SGG, Serviços Gerais de Gestão, S.A., que se dedica a comercializar serviços de Outsourcing nas áreas de contabilidade administrativa e consultoria geral na mesma área.
\par Em Portugal a Deloitte \& Associados, SROC S.A. é responsável por empregar cerca de 1800 colaboradores, divididos por dois Escritórios, Lisboa e Porto. Detêm ainda um escritório em Luanda (Angola) e presta serviços ainda em Cabo Verde, São Tomé e Príncipe e Moçambique.

 % capitulo 1
\chapter{Enquadramento}
\label{cap2}

Este capítulo exemplifica a utilização de referências, figuras, tabelas e listagens.

\section{Projeto}
\par A Brisa solicitou a colaboração da Deloitte para o desenvolvimento de uma plataforma tecnologia que permita disponibilizar informação de gestão consolidada/agregada fazendo uso das suas base de dados de informação de gestão e fazendo uso exclusivo de tecnologias open-source. Desta forma, a Deloitte iniciou uma prestação de serviços com uma duração de 6 meses com a participação de uma equipa de dois recursos, com a minha participação durante três meses e com a supervisão de um gestor de projeto com vasta experiência em soluções de Business Intelligence. A equipa de projeto foi completada com a participação de dois gestores de projeto da Brisa. O planeamento do projeto foi definido fazendo uso de metodologias Agile por diferentes motivos, nomeadamente, a necessidade de concretizar ajuste rápidos ao nível tecnológicos e de arquitetura até a sua estabilização final e devido ao fato de estarmos a colaborar com o cliente na definição de indicadores e layout de dasboards e reports que permitam caraterizar o negócio da melhor forma.
\section{Objetivo}
\par No âmbito do estágio foi definido que a minha participação no projeto iria contemplar todas as fases e componentes do projeto, por forma, a ter uma perceção global e integrada de toda a plataforma. Esta abordagem permitiu ter a experiência de interagir diretamente com o cliente incluindo nas fases de definição de requisitos e de testes de aceitação. 
\par Como objectivo complementar ao projeto, mas não menos importante fui inserido numa equipa de desenvolvimento, com metodologias adequadas mercado de trabalho, contribuindo isto para o aumento das importantes soft-skils.
\section{Planeamento}
\par O projeto foi desenvolvido, como já referido, recorrendo a metodologias Agile. Foi distribuído por quatro fases de desenvolvimento distintas, com uma sequência de trabalhos associada a cada fase. Face ao planeamento inicial ocorreram alguns ajustes no decorrer do estágio para fazer face aos ajustamentos que o projeto careceu durante a sua fase de desenvolvimento, como podemos observar na Figura  \ref{fig:planeamentoFig}. 
\begin{figure}[!htb]
\centering
\includegraphics[width=9cm]{planeamento}
\caption{Planeamento}
\label{fig:planeamentoFig}
\end{figure}
\chapter{Âmbito}
\label{cap3}

\section{Âmbito Funcional}
\par A Brisa é detentora de um Data Mart com as informações relativas à receita e ao tráfego, onde outras ferramentas de reporting estão assentes, entre elas o QlikView. Desta forma, a fonte de dados para a informação de gestão que será disponibilizada no âmbito desta iniciativa será obtida exclusivamente deste Data Mart. 
\par A informação de gestão que será disponibilizada no âmbito desta iniciativa tem como intuito permitir aos responsáveis/utilizadores da Brisa percepcionarem a evolução da receita de tráfego e violações da Brisa, em concreto, permitir de uma forma visual e intuitiva percepcionar as diferenças entre receita teórica e a receita final e violações. Os utilizadores deverão poder percepcionar as diferentes rubricas operacionais que condicionam a receita teórica até a receita final, a título de exemplo: Anulações, Descontos, Isenções e outros.
Os indicadores a disponibilizar são os seguintes, segregados por tipologia:
\begin{itemize}  
	\item Receita Teórica: valor expectável que deveria ser obtido decorrente de transacções obtidas nas diferentes concessões da Brisa. Este valor deverá ser expurgado de violações, isenções e transacções de manutenção.
	\begin{itemize}
	\item Por Barreira: receita teórica para transacções com percursos não válidos. Transacções cujo local de início é desconhecido ou não válido dentro da rede rodoviária.
	\item Por Sublanço: receita teórica para transacções com percursos válidos.  
	\end{itemize}
	\item Ajustamentos: soma de todos os indicadores de ajustamentos que passo a enumerar:
	\begin{itemize}
	\item Anulações: receita efectiva decorrente de transacções que foram anuladas por diferentes motivos;
	\item Descontos: receita efectiva decorrente de transacções onde foram aplicados descontos;
	\item Isenções: receita efectiva decorrente de transacções que não geram receita, a título de exemplo, transacções de veículos de emergência médica, veículos das forças armadas portuguesas, outros. 
	\item Alterações para Violações: receita efectiva decorrente de transacções que foram reclassificadas como violações.
	\item	Alterações de Classe: receita efectiva decorrente de transacções que sofreram alterações de classe dos veículos;
	\item Alterações de Percurso: receita efectiva decorrente de transacções que sofreram alterações de percurso;
	\item Acertos Barreira/Sublanço: receita efectiva decorrente de acertos de barreira e de sublanço. Convém salientar que consideramos que este indicador será calculado considerando os tipos de movimento não incluídos nos indicadores de ajustamento definidos anteriormente.
	\end{itemize}
	\item Violações: soma dos indicadores de violações que passo a enumerar:
	\begin{itemize}
	\item Violações Iniciais Sem Percurso Válido: receita de transacções decorrentes de violações iniciais sem percurso válido;
	\item Violações Iniciais Com Percurso Válido: receita de transacções decorrentes de violações iniciais com percurso válido;
	\item Violações Alterações Sem Percurso Válido: receita de transacções que sofreram alterações relativas a violações sem percurso válido;
	\item Violações Alterações Sem Percurso Válido: receita de transacções que sofreram alterações relativas a violações com percurso válido.
	\item Receita Final: soma dos indicadores receita teórica e ajustamentos.
	\end{itemize}
\end{itemize}
   
\par A par disto, pretendia-se ainda a reformulação de alguns relatórios de QlikView,  bem como o desenvolvimentos de outros não existentes, de modo a complementar a informação nesta ferramenta de reporting.

\section{Âmbito Tecnologico}
\par Após uma breve introdução, apresentam-se as tecnologias utilizadas e o percurso de aprendizagem do autor para utilização das mesmas.

\subsection{Introdução}
\par Inicialmente foi disponibilizado todo o projecto da aplicação WEB, bem como alguns exemplos de utilização de AngularJS e da biblioteca javascript D3.js. Foi ainda disponibilizado o \textit{Data Mart} que continha toda a informação necessária para a implementação do projeto. 
\par Na fase inicial foi crucial o enquadramento com as varias tecnologias e frameworks, partindo de alguns exemplos disponibilizados, possibilitando assim uma aprendizagem autônoma. Posto isto, e após algum estudo do código disponibilizado,  ficaram criadas as condições para a primeira abordagem às tecnologias que iriam ser necessárias dominar, iniciando pela implementação de algumas funcionalidades mais básicas. 

\subsection{Tecnologias utilizadas}
\begin{description}
  \item[JavaEE] O Java Entreprise Edition é a plataforma de desenvolvimento orientada Web recorrendo à linguagem de programação Java. O JavaEE estende do tradicional JavaSE, e disponibiliza várias API’s direccionadas para o desenvolvimento no lado do servidor.
  \item[WebServices] É uma solução utilizada para realizar a comunicação entre diferentes aplicações. Através desta tecnologia é possível que novas aplicações comuniquem com as existentes, e garantido que sistemas desenvolvidos em plataformas diferentes sejam compatíveis. 
\par Cada aplicação pode ter a sua própria linguagem, que é traduzida numa linguagem universal, em formato XML,  Json ou CSV.
  \item[Rest] Permite a troca de informação entre aplicações , através do protocolo HTTP, recorrendo a solicitações GET, POST, DELETE e PUT.
  \item[Spring MVC Framework] O Spring MVC é uma framework open source que visa ajudar no desenvolvimento de aplicações com o JavaEE assente no modelo MVC (Model-View-Controller).
  \item[Sql Server] É um Sistema de Gestão de Base de Dados, vocacionado para base de dados relacionais, desenvolvido pela Microsoft. Actualmente é uma das opções no mercado mais robustas no que toca aos SGBD.
  \item[JSON] O JavaScript Object Notation é uma linguagem leve para a troca de dados entre aplicações. Devido à sua simplicidade tem vindo a ser cada vez mais utilizado em detrimento do XML, em pedidos AJAX. 
\par Apesar de ser uma linguagem derivada do javascript, já muitas outras linguagens suportam o JSON.
  \item[AngularJS] É uma framework javascript para desenvolvimento de aplicações WEB, assente no modelo de MVC. O AngularJS permite o desenvolvimento de aplicações extremamente dinâmicas através da ponte entre o HTML e o javascript. 
  \item[Hibernate ORM] É uma framework Java open source utilizada para facilitar o mapeamento entre atributos de uma base de dados relacional e o modelo de objecto da aplicação Java, através de ficheiros XML ou por uma sintaxe de anotações presente no próprio modelo.
  \item[D3.js] O D3.js é uma biblioteca javascript utilizada para manipular documentos baseados em dados. Esta biblioteca utiliza HTML, SVG e CSS para transformar a fonte de dados em poderosas visualizações, como por exemplo em gráficos.
  \item[QlikView] O QlikView é uma ferramenta de Business Intelligence detentora de uma tecnologia patenteada de associação em memória. O QlikView permite a absorção de dados de várias fontes, entre as quais Excel, XML, SQL Server, Oracle, SAP ou Data Warehouse, podendo ser combinadas as mais variadas fontes de informação, permitindo a consulta e análise dessa informação estruturada em dashboards. O QlikView permite elaborar análises bastantes detalhadas, explorando as relações entre os dados de forma dinâmica.
  \item[JQuery]
  \item[HTML]
  \item[CSS]
  \item[Bootstrap]
  \item[SVN]
  
\end{description}

%\dirtree{%
%.1 debug.
%.2 filename.
%.2 modules.
%.3 module.
%.3 module.
%.3 module.
%.2 level.
%}
\chapter{Aprendizagem}
\label{cap4}
\par Neste capitulo serão apresentadas alguns exemplos realizados antes de iniciar o desenvolvimento do projecto, com o objectivo de adquirir alguns conhecimentos nas tecnologias a utilizar, visto que até à data de inicio do projecto a experiência era reduzida nas mesmas.

\subsection{AngularJS e D3.js}
\par Apesar de já ter uma reduzida experiência com a framework de javascript AngualrJs, não detinha qualquer experiência com a biblioteca de gráficos javascript D3.js. Por esse motivo foi necessário realizar na primeira fase, um pequeno projecto que me permitisse obter alguns conhecimentos acerca desta duas tecnologias, para mais tarde aplicar esses mesmos conhecimentos no projeto. 

\begin{lstlisting}[language=html,caption={Primeiro exemplo em AngularJS e D3.js.},{label=example:angular}]
<!DOCTYPE html>
<html>
<head>
    <meta charset=utf-8 />
    <script src="/js/angular.min.js"></script>
    <script src="/js/d3.js"></script>
    <script src="/js/d3plus.js"></script>
    <link href="/css/bootstrap.min.css" rel="stylesheet" type="text/css" />
    <script src="/js/main_controller.js"></script>
    <title>Angular JS Demo</title>
</head>
<body ng-app="start" ng-controller="mainController" class="text-center">
    <h2 ng-bind="title"></h2>
    <div class="col-lg-6 col-md-6">
        <pie-chart id="viz" data="data" width="400" height="400"></pie-chart>
    </div>
    <div class="col-lg-6 col-md-6">
        <div class="input-group" ng-repeat="data in data">
            <span class="input-group-addon" id="sizing-addon2" ng-bind="data.name"></span>
            <input type="number" class="form-control" placeholder="{{data.name}}" aria-describedby="sizing-addon2" ng-model="data.value">{{data.value}}
        </div>
    </div>
</body>
</html>
\end{lstlisting}

\begin{lstlisting}[language=html,caption={Primeiro exemplo em AngularJS e D3.js.},{label=example:angular}]
(function () {
    var app = angular.module('start', []);
    home.directive('pieChart', function ($compile) {
        return {
            restrict: 'E',
            scope: {
                data: '=?',
                width: '=?',
                height: '=?',
            },
            compile: function (element, attributes) {
                if (!attributes.height) {
                    attributes.height = 200;
                }
                post: function postLink($scope, element /*, attributes*/ ) {
                    $scope.$watch('data', function (newVals, oldVals) {
                        updatePie($scope, element);
                    }, true);
                }
                return postLink;
            }
        }
    });

    app.controller('mainController', ['$scope', function ($scope) {
        $scope.title = "Primeiro exemplo D3.js/AngularJS";
        $scope.data = [
            {"value": 100, "name": "alpha"},
            {"value": 70, "name": "beta"},
            {"value": 40, "name": "gamma"},
            {"value": 15, "name": "delta"},
            {"value": 5, "name": "epsilon"},
            {"value": 1, "name": "zeta"}
        ];
    }]);

    function updatePie($scope, element) {
        d3plus.viz()
            .container("#" + element[0].id)
            .data($scope.data)
            .type("pie")
            .id("name")
            .size("value")
            .width($scope.width)
            .height($scope.height)
            .draw()
    }
})();
\end{lstlisting}


\subsection{Primeira aplicação em \textit{QlikView}}
\par No que respeita ao QlikView foi uma experiência completamente nova, pois até à data de inicio do projecto não tinha qualquer experiência, desconhecendo mesmo esta ferramenta de reporting. Como tal foi necessário despender algum tempo para investigação, e aprendizagem autónoma, de forma a conseguir desenvolver uma pequena aplicação de teste, que me serviu de ponto de partida. O contributo do cliente foi muito importante ao disponibilizar uma aplicação já existente, pois através da sua exploração consegui adquirir os conhecimentos básicos para avançar nesta tecnologia.

% para adicionar o  capítulo N adicione a linha \input{capituloN} e crie o ficheiro 
% capituloN.tex na directoria "capitulos" 

% Bibliografia
%http://www.ieee.org/publications_standards/publications/authors/author_templates.html
\bibliographystyle{bibstyle-ipbeja}  
\bibliography{bibliografiaDoRelatorio}

%%%%%%%%%%%%%%%%%%%%%%%%%%%%%%%%%%%%%%%%%%%%%%%%
\apendices
\chapter{Título do Apêndice I}
\label{ap1}

%a linha seguinte deve ser substituída pelo texto do apêndice
\lipsum
% para adicionar o  apêndice N adicione a linha \input{apendiceN} e crie o ficheiro 
% apendiceN.tex na directoria "apendices" 

\anexos
\chapter{Título do Anexo I}
\label{an1}

%a linha seguinte deve ser substituída pelo texto do anexo
\lipsum
% para adicionar o  anexo N adicione a linha \input{anexoN} e crie o ficheiro 
% anexoN.tex na directoria "anexos" 

\end{document}