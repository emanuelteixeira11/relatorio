\chapter{Introdução}
\label{intro}

\section{Âmbito}

\par O estágio foi concretizado na empresa de auditoria e consultoria Deloitte, nas instalações da Brisa, cliente da Deloitte, inserido na unidade curricular de Estágio ou Projecto, integrante do plano de estudos do curso de Licenciatura em Engenharia Informática da ESTIG, estabelecimento de ensino pertencente ao IPBeja. O trabalho concretizado no estágio contém um nível de dificuldade e complexidade adequado ao mercado de trabalho atual e face as funções que se pretende que um recém-licenciado na área da engenharia informática realize numa empresa de consultadoria. Convém reforçar ainda que as tecnologias utilizadas na concretização do estágio podem atualmente ser consideradas como “state-of-the-art” dado que se tratam de tecnologias muito recentes. O estágio realizado obedeceu aos seguintes pressupostos que tinham sido previamente estabelecidas com os responsáveis do IpBeja e da instituição de acolhimento:
\vspace{-0mm}
\begin{itemize}  \itemsep1pt \parskip0pt \parsep0pt
 \item Duração e Carga Horária: o estágio de duração mínima de três meses, com uma carga horária que se assemelhasse ao verdadeiro mercado de trabalho, cerca de 8 horas diárias, totalizando um total de quarenta horas semanais;
 \item Orientação curricular: Ser orientado por uma figura de mérito reconhecido dentro da instituição, bem como um docente pertencente à ESTIG com o intuito de supervisionar o trabalho realizado na instituição de acolhimento. Na instituição de acolhimento pude contar com o Eng.º Nuno Santos, Senior Manager da área de Consulting da linha de serviço de Application Management Services. Convém salientar que o trabalho operacional diário foi ainda supervisionado por um recurso sénior da instituição, em concreto, pelo team leader Mário Pereira, de acolhimento e pelos responsáveis da Brisa.  A Professora Doutora Isabel Sofia foi a docente do IpBeja responsável por acompanhar o meu trabalho.
\item Aprovação Prévia: obtenção prévia da aprovação do conteúdo e planeamento do estágio a que me propunha, por parte da Comissão de Estágios da LEI. A aprovação foi obtida na data DD-MM-YYYY.
\end{itemize}

\section{Instituição de acolhimento}

\par “Deloitte” é a marca sob a qual dezenas de milhares de profissionais, trabalhando em firmas independentes espalhadas por todo o mundo, colaboram na prestação de serviços de auditoria, consultoria de negócios e de gestão, corporate finance e gestão do risco, consultoria fiscal e serviços relacionados a clientes nos mais diversos setores de atividade.
\par A Deloitte surgiu em 1845 em Londres, sendo hoje uma das maiores empresas no mundo no seus sectores de atividade.
\par A marca “Deloitte”, surge em 1845 em Londres, e atualmente está espalhada pelo mundo, estando presente em mais de 150 países, com cerca de 700 escritórios e mais de 210.000 (duzentos e dez mil) colaboradores. Sobre a marca “Deloitte” operam firmas independentes, cada firma presta serviços numa determinada área geográfica e está restrita à legislação dessa mesma área geográfica onde opera.
\par Em Portugal a entidade membro da Deloitte é a Deloitte \& Associados, SROC S.A. Desta entidade legal fazem parte três subsidiarias que operam em ramos de negócio distinto, sendo que este projecto foi realizado ao serviço da SGG, Serviços Gerais de Gestão, S.A., que se dedica a comercializar serviços de Outsourcing nas áreas de contabilidade administrativa e consultoria geral, incluindo serviços de application management services,  na mesma área.
\par Em Portugal a Deloitte \& Associados, SROC S.A. é responsável por empregar mais  2000 colaboradores, divididos por dois Escritórios, Lisboa e Porto. Detêm ainda dois escritório em Luanda (Angola) e presta serviços ainda em Cabo Verde, São Tomé e Príncipe e Moçambique. Saliento ainda que dada a presença mundial da Deloitte e normal os seus colaboradores participarem em projetos internacionais por todo o mundo.


\section{Estrutura do documento}

