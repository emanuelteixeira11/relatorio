\chapter{Resumo}
\section*{\textit{\TITULO}\\  {\small{\textit{\SUBTITULO}}}}

\par O presente documento tem como intuito descrever o trabalho realizado no âmbito da unidade curricular de Estágio e Projeto da Licenciatura de Engenharia Informática do Instituto Politécnico de Beja, no decurso do ano letivo de 2014/2015. Para a concretização do estágio fui integrado numa equipa de projeto da Deloitte num grupo nacional do sector rodoviário português e internacional, a Brisa. O projeto consiste na disponibilização de informação de gestão e operacional no portal da Brisa fazendo uso exclusivo de tecnologias open-source nas componentes de reporting por forma a minimizar custos financeiros. O intuito do trabalho é potenciar a utilização do portal da Brisa e dotar a Brisa de um conjunto de informação de gestão que permita percecionar, potenciar e alavancar o negócio. O trabalho realizado abrangeu todas as componentes da arquitectura definida no decurso do projeto. Em concreto, consistiu na construção de dashboards e gráficos sobre informações de gestão sobre receitas de tráfego, afetas ao ramo de negócio da empresa Brisa O\&M. Esta informação será disponibilizada sob a forma de widgets para posteriormente serem incluídos no portal corporativo já existente.
\par A implementação destas funcionalidades foi efetuada através de tecnologias totalmente open source, fazendo com que esta informação chegue a um número considerável de utilizadores, sem aumentar os custos de licenciamento inerentes a sistemas de reporting atualmente em uso na empresa. Paralelamente a informação dos dashboards foi complementada com a inclusão de novas métricas e indicadores atuais recorrendo a ferramenta de reporting QlikView. A informação disponível no QlikView permite o acesso a informação de maior detalhe, contudo menos acessível, devido aos custos de licenciamento da ferramenta.
\par Para além das tarefas supracitadas foi também desenvolvido um relatório em formato de newsletter na aplicação de report QlikView, e todo o processo de desenvolvimento estará descrito no presente documento.
\newline


\textbf{Palavras-chave}: \textit{JavaScript, D3.js, AngularJS, JAVA, QlikView}.